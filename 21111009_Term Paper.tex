\documentclass[12pt]{article}
\usepackage[utf8]{inputenc}
\usepackage{authblk}
\usepackage[dvipsnames]{xcolor}
\usepackage{graphicx}

\title{\textbf{\textcolor{PineGreen}{\underline{A REPORT ON MUSIC THERAPY}}}}

\author{\textbf{\textcolor{Blue}{BY AMARRAM MADHU MALVEETIL}}}
\affil[]{\textcolor{Blue}{\textbf{ROll NO.:21111009}}}
\affil[]{\textbf{\textcolor{Brown}{BIOMEDICAL ENGINEERING}}}
\affil[]{\textbf{\textcolor{RedViolet}{"NATIONAL INSTITUTE OF TECHNOLOGY, RAIPUR", CHATTISGARH"}}}
\affil[]{\textbf{\textcolor{Blue}{BATCH:2025\hspace{2cm}SEMESTER:I}}}
\affil[]{\textbf{\textcolor{Maroon}{Under the supervision of}}}
\affil[]{\textbf{\textcolor{Maroon}{Dr. Saurabh Gupta}}}
\affil[]{\textbf{\textcolor{Maroon}{DEPARTMENT OF BIOMEDICAL ENGINEERING}}}
\date{\textbf{\textcolor{Blue}{SUBMITTED ON APRIL 08, 2022}}}
\begin{figure}
    \centering
    \includegraphics[height=5cm, width=5cm]{nit raipur.jpg}
\end{figure}

\begin{document}


\maketitle
\newpage
\section*{\textbf{\hspace{1cm}\textcolor{red}{\underline{\huge{ACKNOWLEDGEMENT}}}}}
\hspace{1cm}\Large{\emph{On the successful completion of my Term Paper, I would like to extend my sincere and heartfelt gratitude to all the people who were involved in the project and supported me along this wonderful experience. Without their solemn efforts and guidance it wouldn’t have been possible for me to complete the work well before the said deadline.}}\vspace{0.5cm}\newline  

\hspace{1cm}\Large{\emph{I would like to sincerely thank Dr. Saurabh Gupta Sir, Department Of Biomedical Engineering, for his consistent guidance and support in the purview of Biomedical Engineered Innovations and also for entrusting me with Term Paper research. Also, I would like to thank our college, “National Institute Of Technology, Raipur” for providing us the right platform and opportunities that are essential for our overall growth and self-development.}}\vspace{0.5cm}\newline 

\hspace{1cm}\Large{\emph{Also, I would like to thank my parents for blessing me with the strength to gain enough enterprise about my topic and complete the project within the stipulated time. I am really grateful to everyone.}} 
\newpage

\section*{\textbf{\hspace{1cm}\textcolor{red}{\underline{\huge{ABSTRACT}}}}}
\hspace{1cm}\Large{\emph{Music therapy is an evidence-based and  tremendously effective methodology that enables the treatment of a variety of disorders including cardiac conditions, depression, autism, substance abuse and Alzheimer's disease. etc. It can even help with memory, lower blood pressure, improve coping, reduce stress, improve self-esteem, boost concentration power and many more. With exponential development of technologies and infrastructure, both patients and therapists are, today, benefiting from technological applications and the massive influence they have on our lives. This research reviews the significance of music therapy in the physiological and psychological treatment of patients suffering from serious illnesses and non-benefited by the already present practises. The numerous facets of Biomedical acoustics and music therapy will be scrupulously dealt with and the current challenges and loopholes in this regard will also be discussed. Special emphasis shall be made on the impact of music therapy on mental well-being and the role of alpha music-waves in dealing with depression cases.}}

\section*{\textbf{\hspace{1cm}\textcolor{red}{\underline{\huge{INTRODUCTION}}}}}
\hspace{1cm}\Large{\emph{Music therapy is the clinical and evidence-based implementation of music interventions to accomplish individualized goals within a therapeutic relationship by a certified professional who has been accredited with a music therapy license. Music is that aspect of nature which touches all aspects of mind, body, brain and behaviour.It can soothe our mind, clean all the mental dirt, drive us forth emotionally towards a positive direction, placate the rhythms of our body, alter our mood and what not...In short, Music Therapy is the well-defined use of music to address the physical, emotional, cognitive, and social needs of a group or an individual in particular. A variety of activities, such as listening to melodies, playing instruments,writing songs, and guided imagery, etc, are employed. Music therapy is practically appropriate for people of all ages and kinds, whether they are tone deaf, struggling with any impairments or totally healthy; any way it's beneficial.}}\vspace{1cm}\newline
\hspace{1cm}\Large{\emph{Over the past 2 decades, Music Therapy has found its way through Neurosciences, by hinting towards the possibilities of brain treatment at a much subtler level. Biomedical researchers have found that music is a highly structured and patternised auditory language involving complex perception, cognition, and motor control in the brain, and hence; it can effectively be used to retrain and revamp the injured brain. While the initial data giving these results were met with great skepticism and even resistance, over the years, the consistent accumulation and aggregation of scientific and clinical research evidences have diminished the doubts to a great extent. Therapists and physicians nowadays use music in rehabilitation in various ways that are not only backed up by clinical research findings but also supported by the comprehensive  understanding of some of the mechanisms of music and brain functions.}}
\newpage

\subsection*{\textbf{\hspace{1cm}\textcolor{red}{\huge{1.\underline{ROLE OF NEUROSCIENCE}}}}}
\hspace{1cm}\Large{\emph{New brain imaging and electrical recording techniques have combined to transform our understanding of music in therapy and education over the last two decades. For the first time, we were able to observe the living human brain while people performed complicated cognitive and motor activities using these techniques (functional magnetic resonance imaging, positron-emission tomography, electroencephalography, and magnetoencephalography). It was now possible to perform perception and cognition studies in the arts.}}\vspace{1cm}\newline
\hspace{1cm}\Large{\emph{Music has been a part of imaging research since the very beginning. It was used as a model to explore how the brain interprets verbal versus nonverbal communication, how it processes complex time information, and how the brain of a musician enables the advanced and complex physical abilities required to produce a musical composition.}}\vspace{1cm}\newline
\hspace{1cm}\Large{\emph{Therapists could finally build a powerful, testable hypothesis for using music in rehabilitation by combining these developments—brain imaging, insight into plasticity, and the discovery that musical and non-musical functions share systems—by combining these developments—brain imaging, insight into plasticity, and the discovery that musical and non-musical functions share systems. Through common brain circuits and plasticity, music can drive broad reeducation of cognitive, motor, and speech and language capabilities. Music could now be examined as a potential aspect of active learning and training, rather than only as an additional stimulant to support therapy.}}\vspace{1cm}

\subsection*{\textbf{\hspace{1cm}\textcolor{red}{\large{2.\underline{MUSIC THERAPY FOR PARKINSON'S TREATMENT}}}}}
\hspace{1cm}\Large{\emph{Music therapy addresses issues that typically impact persons with PD, such as bradykinesia(a slowness of movement that can lead to difficulties with activities of daily living), by using rhythm, melody, and preferred movement.}}\vspace{1cm}\newline
\hspace{1cm}\Large{\emph{Rhythm becomes a template for organizing a series of movements, as well as combat cognitive issues that affect movement function, such as attention and focus. Rhythm helps coordinate movement, stimulate attention spans and induce relaxation. When a PD patient perceives these rhythmic flows, he/she can actually feel the positive outcomes.}}\vspace{1cm}
\subsection*{\textbf{\hspace{1cm}\textcolor{red}{\large{3.\underline{ALPHA MUSIC}}}}}
\hspace{1cm}\Large{\emph{Alpha brain waves are a type of electrical activity that the brain produces. The brain is made up of millions of neurons that communicate using electrical signals. When one is daydreaming, meditating, or practising mindfulness, he/she is likely to experience alpha waves. According to research, this sort of brain wave may aid in the reduction of depression symptoms and the enhancement of creativity.}}\vspace{1cm}\newline
\hspace{1cm}\Large{\emph{Recent findings have proven that a technique called "Transcranial Alternating Current Stimulation(TACS)" induces alpha wave activity and reduces depressive symptoms in people with major depressive disorder.Due to this, these people have now been able to overcome overthinking habits and divert their vital energies more towards creative activities.}}\vspace{1cm}\newline
\hspace{1cm}\Large{\emph{Another study suggested that increasing both alpha and theta activity in the brain’s occipital lobes helped decrease anxiety and improve functioning in those with generalized anxiety disorder.}}

\section*{\textbf{\hspace{1cm}\textcolor{red}{\underline{\Large{FUTURE SCOPE AND CONCLUSION}}}}}
\hspace{1cm}\Large{\emph{The auditory scaffolding theory and the extended shared brain system theory offered a new theoretical foundation for the therapeutic use of music in motor, speech, and language rehabilitation, as well as cognitive rehabilitation. New theories may one day aid our understanding of music's other impacts and show the way to new sorts of rehabilitation.}}\vspace{1cm}\newline
\hspace{1cm}\Large{\emph{Biomedical research in music has come a long way in allowing music to reeducate the wounded brain in new and effective ways. Of course, there is still more to be done: Scientists need to learn more about what dosages work best, focus on studies that will benefit children, and concentrate on conditions where neurologic music therapy has yet to be thoroughly studied, such as autism, spinal cord injury, cerebral palsy, and multiple sclerosis. Furthermore, because the effects of brain injury can be complicated, researchers must account for individual characteristics and adjust to individual demands. Those goals are shared by neurologic music therapists and practitioners in other therapeutic fields.}}\vspace{1cm}\newline
\hspace{1cm}\Large{\emph{Although neurologic music therapy is a specialised field, it is founded on aspects and principles of music and brain function that may be applied by anyone in the rehabilitation field. As a result, it provides a solid platform for patient-centered interdisciplinary collaboration.}}

\section*{\textbf{\hspace{1cm}\textcolor{red}{\underline{\Large{REFERENCES}}}}}
\vspace{0.2cm}
\subsection*{\textbf{1.\hspace{1cm}\large{\textcolor{blue}{ https://dl.acm.org/doi/10.1145/3403782.3403789}}}}
\subsection*{\textbf{2.\hspace{1cm}\large{\textcolor{blue}{	https://my.clevelandclinic.org/health/treatments/8817-music-therapy}}}}
\subsection*{\textbf{3.\hspace{1cm}\large{\textcolor{blue}{	https://en.wikipedia.org/wiki/Music_therapy}}}}
\subsection*{\textbf{4.\hspace{1cm}\large{\textcolor{blue}{	https://www.takingcharge.csh.umn.edu/common-questions/what-music-therapy}}}}
\subsection*{\textbf{5.\hspace{1cm}\large{\textcolor{blue}{	https://dana.org/article/how-music-helps-to-heal-the-injured-brain/}}}}


\end{document}
