\documentclass[12pt]{article}
\usepackage[utf8]{inputenc}
\usepackage{authblk}
\usepackage[dvipsnames]{xcolor}
\usepackage{graphicx}

\title{\textbf{\textcolor{PineGreen}{\underline{"Solutions to Covid-19 by Biomed.Engineers"}}}}
\author{\textbf{\textcolor{Blue}{BY AMARRAM MADHU MALVEETIL}}}
\affil[]{\textcolor{Blue}{\textbf{ROLL NO.:21111009}}}
\affil[]{\textbf{\textcolor{Brown}{"Department of BIOMEDICAL ENGINEERING"}}}
\affil[]{\textbf{\textcolor{RedViolet}{"NATIONAL INSTITUTE OF TECHNOLOGY, RAIPUR", CHATTISGARH"}}}
\affil[]{\textbf{\textcolor{Blue}{BATCH:2025\hspace{2cm}SEMESTER:I}}}
\affil[]{\textbf{\textcolor{Maroon}{Assignment 6 of"BASIC BIOMEDICAL ENGINEERING"}}}

\date{\textbf{\textcolor{Blue}{SUBMITTED ON MARCH 4, 2022}}}

\begin{document}
\begin{figure}
    \centering
    \includegraphics[height=5cm, width=5cm]{nit raipur.jpg}
\end{figure}

\maketitle

\section*{\textbf{1.\hspace{1cm}\textcolor{red}{\underline{\huge{INTRODUCTION}}}}}
\hspace{1cm}\large{\emph{The worldwide Covid-19 pandemic was an inevitable headache and one of the most challenging issues the world has ever faced. Many lives had gone during the last 2 treacherous years setting back emotional and mental grief over many families. Humongous healthcare workers and frontiers selflessly sacrificed their lives in this challenging fight. Nevertheless, stringent efforts were taken in restoration of everything and safeguarding human lives with a sophisticated use of technology and science. When it comes to the vaccines and med-tech, the leading role was played by \textbf{Biomedical Engineers}.}}\vspace{1cm}\newline
\hspace{1cm}\large{\emph{The Biomedical Engineers certainly acted as the leading frontiers by improving the healthcare with their technological services. Ranging from Vaccines to new machines to hospital assistance various aspects were taken care by these \textbf{messengers of God}. Throughout this dissertation, let's take an insightful journey to some of the legit and impactful solutions given by the Biomedical Engineers.}}
\newpage

\subsection*{\textbf{\hspace{1cm}1.1.\hspace{1cm}\textcolor{red}{\underline{\Large{VITAL SIGNS MONITOR}}}}}

\hspace{1cm}\large{\emph{It has now become mandatory to monitor the  vital signs of COVID patients during illness as well as during the recovery stage. Vital sign monitors constantly check blood-oxygen saturation, heart condition, breathing and movement of the patient.  Since COVID virus targets the lungs, it becomes obligatory to monitor the Blood Oxygen Saturation (SpO2) also known as \textbf{“peripheral capillary oxygen saturation”.}}}\vspace{1cm}\newline
\hspace{1cm}\large{\emph{Wearable Sensors have already been initiated to the western market to enable comfortability and mobility of patient. Even when the patient is at home, he/she can have peace of mind by tracking the SpO2 levels using these wearable sensors. This is undoubtedly a genuine research that has successfully impacted the lives of many people.}}

\begin{figure}
    \centering
    \includegraphics[height=05cm, width=09cm]{wearable sensor.png}
\end{figure}

\newpage

\subsection*{\textbf{\hspace{1cm}1.2.\hspace{1cm}\textcolor{red}{\underline{\Large{PERSONAL PROTECTIVE EQUIPMENT}}}}}

\hspace{1cm}\large{\emph{During COVID it had become very crucial to get personal protective equipments like Virus-resistant face shield, gowns, gloves and masks, arranged in hospitals for the safety of healthcare workers.Personal protective devices such as protective clothing was designed and made using surgical drapes and plastics.  Diagnostic testing booths equipped with HEPA filters were made to reduce the use of surgical gowns and medical supplies. This booth is more comfortable for healthcare personnel and lowers the probability of infection during the removal of gowns. Lots of innovative solutions for face shields were also developed. These range from face shields made using 3D-printed frames and thin plastic binder sleeves, to fashionable consumer face shields.}}

\begin{figure}
    \centering
    \includegraphics[height=05cm, width=09cm]{PPE.png}
\end{figure}
\vspace{3cm}
\newpage

\subsection*{\textbf{\hspace{1cm}1.3.\hspace{1cm}\textcolor{red}{\underline{\Large{TELE-MEDICINE AND REMOTE MONITOR}}}}}

\begin{figure}
    \centering
    \includegraphics[height=05cm, width=09cm]{telemedicine.png}
\end{figure}

\hspace{1cm}\large{\emph{When vaccines weren't made available during the pandemic, activities such as meeting people, gathering with a crowd and even visiting a doctor were serious challenges. Innovations in communication using teleconferences or online consultations have increased significantly. Telemedicine was actually promoted several years ago, but gained little traction. However, during the COVID-19 pandemic, telemedicine now delivers medical care remotely to millions using communications technology. By using more sophisticated technologies like videoconferencing and other new emerging applications, telemedicine will continue to become popular in our health care system.}}\vspace{1cm}\newline
\hspace{1cm}\large{\emph{Telemedicine has facilitated patient monitoring through computer or tablet or phone technology that has reduced outpatient visits. Now doctors can verify prescription or supervise drug oversight. Furthermore, the home-bound patients can seek medical-help without moving to clinic through ambulance. Thus, cost of health care has been reduced. This system also facilitates health education, as the primary level healthcare professionals can observe the working procedure of healthcare-experts in their respective fields and the experts can supervise the works of the novice. Telemedicine also eliminates the possibility of transmitting infectious diseases between patients and healthcare professionals.}}\vspace{1cm}

\subsection*{\textbf{\hspace{1cm}1.4.\hspace{1cm}\textcolor{red}{\underline{\Large{COVID DIAGNOSTIC KIT}}}}}

\hspace{1cm}\large{\emph{In any communicable infection scenario, the most important aspect is to detect who is and who is not infected with the infectious agent as quickly as possible. For COVID-19, it is difficult to detect the level of infection because of its unique way of infecting people.  According to the CDC, the first 1-5 days tends to be asymptomatic. This means that the infected individual does not manifest any symptoms even though the individual is already infected and does not know that they may have started to spread the disease.  After that stage, the person starts to show symptoms and the disease antibodies start to be detectable. The most common symptoms are fever, body aches, difficulty of breathing, and loss of smell among others.To identify and track all infected individuals, it is of prime importance to test practically everybody.}}\vspace{1cm}\newline
\hspace{1cm}\large{\emph{Innovations on diagnostic tools are on-going using different technologies such as new fluorogenic aptamers, nanotechnology, quantum dots, advanced image sensors, electrochemical sensors, microfluidics, and other technologies. Many approved diagnostic tools for detecting COVID-19 using proprietary molecular point-of-care test platforms were deployed during COVID. Most of these diagnostic tools include an assay designed to detect the virus, a microfluidic cartridge, and a reader.}}

\begin{figure}
    \centering
    \includegraphics[height=05cm, width=09cm]{kit.jpg}
\end{figure}
\vspace{1cm}

\subsection*{\textbf{\hspace{1cm}1.5.\hspace{1cm}\textcolor{red}{\underline{\Large{ UV-C  DISINFECTION}}}}}

\hspace{1cm}\large{\emph{Sanitation and disinfection are a primary tool in our healthcare system to prevent the spread of infection.  People want to be sure that their place of work and anything they will touch is safe and free from the virus.  In hospital settings, it is crucial that operating rooms or ICUs are routinely disinfected and sanitized. Any human or procedural error during the disinfection can cause hospital-acquired infections (HAIs).}}\vspace{1cm}\newline

\hspace{1cm}\large{\emph{The most commonly used disinfection tools are liquid and sprayed disinfectants, which can be complicated and burdensome to use effectively. Engineers have explored a new approach using ultraviolet lights in the so-called UV-C range (100-280 nm).  Applications include ICU and patient room disinfection, sanitizing gadgets such as mobile phones (b), personal devices,  air purifiers, and N95 masks for possible reuse. This technique does not require liquid disinfectant, but rather disinfects by exposing the affected area to UV light.}}

\begin{figure}
    \centering
    \includegraphics[height=05cm, width=09cm]{UV-C.png}
\end{figure}
\vspace{1cm}

\end{document}
